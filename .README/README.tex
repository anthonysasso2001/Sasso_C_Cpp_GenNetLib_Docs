\documentclass{article}

%   included packages
\usepackage[a4paper,margin = 14mm]{geometry} %0.5 inch margins a4 paper
\usepackage{amsmath}
\usepackage{newpxtext,newpxmath}
\usepackage{microtype}
\usepackage{titling}

%   footer, to show license information, specifying that data is CC-BY-4.0 BUT non documentation code snippets are GPL3 to unify with main project lisencing.
\usepackage{fancyhdr}
% set page style to fancy reset and disable header and footer, then allocate footer type
\pagestyle{fancy}
\fancyhead{}
\renewcommand{\headrulewidth}{0pt}
\fancyfoot{}
\fancyfoot[R]{\thepage}
\fancyfoot[C]{\textbf{Documentation licensed under CC-BY-4.0, code \& software examples licensed under GNU GPLv3.}}


%   Allows me to use hyperlinks for table of contents etc.
\usepackage{hyperref}
\usepackage{xcolor}
\hypersetup{
    linktoc     =   all,
    colorlinks,
    citecolor   =   black,
    filecolor   =   black,
    linkcolor   =   black,
    urlcolor    =   blue
}

%   Setting up the code examples
% \usepackage{listings}


% \definecolor  {dkgreen}   {rgb}   {0,0.6,0}
% \definecolor  {gray}      {rgb}   {0.5,0.5,0.5}
% \definecolor  {mauve}     {rgb}   {0.58,0,0.82}

% \lstset{
%   frame               =   tb,
%   language            =   C++,
%   aboveskip           =   3mm,
%   belowskip           =   3mm,
%   showstringspaces    =   false,
%   columns             =   flexible,
%   basicstyle          =   {\small\ttfamily},
%   numbers             =   none,
%   numberstyle         =   \tiny\color{gray},
%   keywordstyle        =   \color{blue},
%   commentstyle        =   \color{dkgreen},
%   stringstyle         =   \color{mauve},
%   breaklines          =   true,
%   breakatwhitespace   =   true,
%   tabsize             =   3
% }

%   formatting for document

%   Push everything to the left cause I think it looks cleaner
\preauthor  {\begin{flushleft}}
\postauthor {\end{flushleft}}
\predate    {\begin{flushleft}}
\postdate   {\end{flushleft}}    
\setcounter {secnumdepth}{0}

%   Document Title & author
\title      {\Huge Sasso\_C\_Cpp\_GenNetLib Documentation}
\author     {By: Anthony Sasso}

%   updates date for each compile, that way this + versioning tells reader which they are reading?
\date       {\today}

%   no indents for paragraphs (change depending on doc type)
\setlength  {\parindent}    {0pt}

%   custom commands / formatting shortcuts
%   combine bold & italic
\newcommand{\textbfit}[1]{\textbf{\textit{#1}}} 

%   footnote with label, colour, and inputted text... then sets back to default
% \newcommand{\clfootnote}[3]{\color{#2}{\footnote{\label{#1}{ \color{#2}{#3}}}}\color{defaultcolor}} 

%   set to defualt colour (black)
\AtBeginDocument    {\colorlet{defaultcolor}{.}}

%   writing goes here
\begin{document}

\maketitle

\tableofcontents

\newpage
\section{Introduction}
Separate documentations for Sasso\_C\_Cpp\_GenNetLib intended to be included as a submodule or similar to help segregate the code better.

\section{Getting Started}
\subsection{Software Dependencies}
\begin{enumerate}
    \item Ensure a Text editor like LaTeX or LibreOffice, as well as Some Visualization Program like LibreOffice Draw or Inkscape to allow a unified, accessible format for the documentation Source.
\end{enumerate}

\section{Build \& Test}
\begin{enumerate}
    \item Download / Clone Target Branch (by default would recommend the latest one).
    \item Modify the appropriate documentation file, adding separate documentation as needed.
    \item Review Spelling, format, (if applicable) convert to a PDF, SVG, or PNG version.
    \item Commit code according to contribute steps below.
\end{enumerate}

*Please note all documentation should include the footer \textbf{``Documentation licensed under CC-BY-4.0, code \& software examples licensed under GNU GPLv3.''} for clarification, in the case of visualizations this may be included by a floating box in the corner of the display. Easiest way for TeX documents is to copy this README's code from \% footer, to end of \textbackslash{}fancyfoot[C] (lines 10-18).

\section{Contribute!}
To Contribute to the project follow these rules:
\begin{itemize}
    \item First create an issue / feature request notifying any changes that are wanted.
    \item Wait to hear the initial response / accepting of that issue before you begin coding.
    \item Fork or Branch (dependent on scope, membership in core development team, etc.) the repo and code any necessary fixed to fulfill this new feature / fix the bug.
    \item Issue a pull request back to the main repo with your latest released version and attach a text file or description within the pull request detailing your changes, additions, considerations with this pull request. In addition, follow the ``LatexToMarkdownNoted.md'' within the ``.README'' Source folder to update the README listing any necessary changes, and have this as your final commit before the pull request.
\end{itemize}

\end{document}
